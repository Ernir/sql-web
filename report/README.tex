\section*{SQLweb}\label{sqlweb}

\textbf{(Ísl)} Kennsluvefur í notkun gagnasafna. Hluti af
meistaraverkefni Eiríks Ernis Þorsteinssonar við IVT-deild Háskóla
Íslands. Uppsetningarleiðbeiningar má finna að neðan.

\textbf{(En)} A teaching website for introductory database usage.
Written as a part of Eiríkur Ernir Þorsteinsson's master's project,
Faculty of Industrial Engineering, Mechanical Engineering, and Computer
Science at the University of Iceland. Installation instructions are
currently in Icelandic only, but can be found below.

\subsection*{Uppsetning
þróunarútgáfu}\label{uppsetning-uxferuxf3unaruxfatguxe1fu}

Vefurinn er skrifaður í \href{https://www.djangoproject.com/}{Django} og
ætti að keyranlegur á flestum nútíma stýrikerfum. Þessar
uppsetningarleiðbeiningar gera ráð fyrir
\href{https://www.ubuntu.com/}{Ubuntu} Linux 16.04 eða sambærilegu
kerfi. Gert er ráð fyrir að lesandinn sé kunnugur skipanalínunotkun.

\subsubsection{0. Undanfarar}\label{undanfarar}

Áður en hafist er handa er nauðsynlegt að eftirfarandi sé uppsett:

\begin{itemize}
\item
  \href{https://www.python.org/downloads/}{python 3} með þróunartólum,
  sýndarumhverfakerfi og pakkakerfi (nauðsynlegt)
\item
  þýðingartól sem tengjast C- og Postgresviðbótum
\item
  gagnagrunnskerfi, hér gert ráð fyrir
  \href{https://sqlite.org/}{SQLite}
\end{itemize}

að auki er sterklega mælt með: * \href{https://git-scm.com/}{git} til að
sækja og viðhalda forritskóða * cURL og tar til að sækja myndir

Á Ubuntu má setja þetta allt saman upp með eftirfarandi skipunum:

\begin{minted}[fontsize=\small, frame=lines]{bash}
$ sudo apt-get update && sudo apt-get upgrade
$ sudo apt-get install git python3-dev python3-pip 
    python3-venv sqlite3 build-essential libpq-dev curl
\end{minted}

\subsubsection{1. Stilling á
umhverfisbreytum}\label{stilling-uxe1-umhverfisbreytum}

Keyrsla vefsins krefst þess að tvær umhverfisbreytur (e.
\emph{environment variables}) séu stilltar. Þær eru
\texttt{DEBUG\_MODE}, sem við þróun ætti að vera \texttt{1} og
\texttt{SECRET\_KEY} sem er ekki-tómur strengur.

Á staðaluppsettu Ubuntu er að afgreiða breyturnar með því að setja línur
á borð við eftirfarandi í \texttt{.bashrc} skrána og endurhlaða svo
breytunum (t.d. með því að endurræsa skelina).

\begin{minted}[fontsize=\small, frame=lines]{bash}
export SECRET_KEY="lykill"
export DEBUG_MODE="1"
\end{minted}

Í raunverulegu keyrsluumhverfi ætti \texttt{SECRET\_KEY} að vera langur
strengur sem erfitt er að giska á og \texttt{DEBUG\_MODE} að vera 0.

\subsubsection{2. Forritskóði sóttur}\label{forritskuxf3uxf0i-suxf3ttur}

Til að ná í nýjustu útgáfu af forritskóða vefsins má nota git. Færum
okkur í yfirmöppu þeirrar möppu sem vefurinn skal dvelja í og sækjum
þvínæst forritskóðann með:

\begin{minted}[fontsize=\small, frame=lines]{bash}
$ git clone https://github.com/Ernir/sql-web.git
\end{minted}

en skipunin mun búa til möppu að nafni \texttt{sql-web}. Færum okkur til
hennar og höldum okkur þar þangað til uppsetningu er lokið.

\begin{minted}[fontsize=\small, frame=lines]{bash}
$ cd sql-web/
\end{minted}

\emph{Annar valmöguleiki}: Hægt er að sækja forritskóðann án þess að
nota git með því að fara inn á
\href{https://github.com/Ernir/sql-web}{github-síðu verkefnisins} og
sækja hana sem .zip skrá.

\subsubsection{3. Uppsetning og virkjun
sýndarumhverfis}\label{uppsetning-og-virkjun-suxfdndarumhverfis}

Ráðlagt er að keyra Python-forrit önnur en hin fánýtustu í
sýndarumhverfi (e. \emph{virtual environment}) sem heldur utan um
forritssöfn. Stofnum til nýs sýndarumhverfis að nafni \texttt{sqlvenv}.

\begin{minted}[fontsize=\small, frame=lines]{bash}
$ python3 -m venv sqlvenv
\end{minted}

Áður en umhverfið má nota þarf að virkja það. Að óbreyttu mun nafn
sýndarumhverfisins birtast fremst á skipanalínunni meðan það er virkt.

\begin{minted}[fontsize=\small, frame=lines]{bash}
$ source sqlvenv/bin/activate
(sqlvenv) $ 
\end{minted}

\emph{Athugasemd}: Ekki er þörf á því núna, en til að slökkva á
sýndarumhverfinu má gefa skipunina \texttt{deactivate}.

\begin{minted}[fontsize=\small, frame=lines]{bash}
(sqlvenv) $ deactivate 
$
\end{minted}

\subsubsection{4. Uppsetning pakka sem vefurinn
krefst}\label{uppsetning-pakka-sem-vefurinn-krefst}

Í möppunni \texttt{sql-web} má finna skrá sem heitir
\texttt{requirements.txt}, sem inniheldur upplýsingar um forritssöfn sem
vefnum eru nauðsynleg. Setjum söfnin upp með \texttt{pip}
pakkastjórnunarkerfinu.

Í þessu skrefi er mikilvægt að sýndarumhverfið sé virkt!

\begin{minted}[fontsize=\small, frame=lines]{bash}
(sqlvenv) $ pip install -r requirements.txt
\end{minted}

\subsubsection{5. Uppsetning gagna}\label{uppsetning-gagna}

Sjálfgefin gögn eru til fyrir kennsluvefinn, sem innihalda mögulega
upphafsuppsetningu á greinum og sýnidæmi um verkefni. Þau má finna í
SQLite-gagnagrunnsskránni \texttt{sqlweb.db} sem sækja má með

\begin{minted}[fontsize=\small, frame=lines]{bash}
(sqlvenv) curl -O https://notendur.hi.is/~ernir/sql-web/sqlweb.db
\end{minted}

Sjálfgefnu gögnin innihalda vísanir í myndir. Myndirnar eiga heima í
skrá sem heitir \texttt{mediafiles} og er undirmappa \texttt{sql-web}
möppunnar. Eftirfarandi skipun (sem krefst cURL og tar) sækir myndirnar
og setur þær í viðeigandi möppu:

\begin{minted}[fontsize=\small, frame=lines]{bash}
(sqlvenv) $ curl https://notendur.hi.is/~ernir/sql-web/mediafiles.tar.gz | tar -xz
\end{minted}

Eigi ekki að nota sjálfgefnu gögnin má upphafsstilla tóman gagnagrunn
fyrir vefinn með skipuninni

\begin{minted}[fontsize=\small, frame=lines]{bash}
(sqlvenv) $ python manage.py migrate
\end{minted}

\emph{Annar valmöguleiki}: Hægt er að komast hjá því að nota cURL og/eða
tar. Þá þarf að vista
\href{https://notendur.hi.is/~ernir/sql-web/sqlweb.db}{.db skrána} í
grunnmöppunni. Til að fá myndirnar þyrfti að búa til \texttt{mediafiles}
möppuna handvirkt og sækja þær sem
\href{https://notendur.hi.is/~ernir/sql-web/mediafiles.zip}{.zip skrá}.
Þetta gæti verið viðeigandi fyrir Microsoft Windows notendur.


\subsubsection{6. Uppsetning ofurnotandareiknings
(valkvæmt)}\label{uppsetning-ofurnotandareiknings-valkvuxe6mt}

Setja þarf upp ofurnotanda fyrir vefinn ef gera á efnisbreytingar. Það
má gera með skipuninni

\begin{minted}[fontsize=\small, frame=lines]{bash}
(sqlvenv) $ python manage.py createsuperuser
\end{minted}

og fylgja leiðbeiningunum sem birtast á skjánum.

\subsubsection{7. Keyrsla
þróunarvefþjóns}\label{keyrsla-uxferuxf3unarvefuxfejuxf3ns}

Með Django fylgir vefþjónn sem er hentugur til keyrslu á þróunarvélum. Á
honum má kveikja með skipuninni

\begin{minted}[fontsize=\small, frame=lines]{bash}
(sqlvenv) $ python manage.py runserver
\end{minted}

og heimsækja svo síðuna með því að fara á slóðina
\texttt{http://127.0.0.1:8000/} í vafra.

Til að breyta eða bæta við námskeiðum, lesefni eða verkefnum má fara inn
á slóðina \texttt{http://127.0.0.1:8000/admin} og skrá sig inn með
ofurnotendaaðganginum sem búinn var til í skrefi 5.

\subsection*{Uppsetning
keyrsluútgáfu}\label{uppsetning-keyrsluuxfatguxe1fu}

Uppsetning keyrsluútgáfu (til opinberrar birtingar) er í flestum atriðum
eins og uppsetning þróunarútgáfu. Af öryggisástæðum þarf þó að gera
breytingar á umhverfisbreytum, sjá skref 1 í uppsetningu þróunarútgáfu.

\begin{minted}[fontsize=\small, frame=lines]{bash}
export SECRET_KEY="miklulengristrengurenþetta"
export DEBUG_MODE="0"
\end{minted}

Auk þess þarf að setja upp afkastameiri og öruggari vefþjón en þróunarvefþjóninn sem er innbyggður í Django.
Þekkt er að \href{http://gunicorn.org/}{Gunicorn} með \href{https://www.nginx.com/}{NGINX} hentar vel.